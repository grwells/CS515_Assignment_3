\documentclass[a4paper,11pt]{article}
\usepackage[margin=1in]{geometry}
\usepackage{graphicx}
\usepackage{caption}
\usepackage{subcaption}

\title{3: Project Write-up}
\author{
Jacob Friedberg \\
\texttt{frie2142@vandals.uidaho.edu}
\and
Garrett Wells \\
\texttt{well1157@vandals.uidaho.edu}
}


\begin{document}
\maketitle
\begin{abstract}
    This project used techniques presented in the course to estimate parameters for a Hidden Markov Model of the provided data set. We prepared a substitution matrix, emission probability table, and state transition table from the data. The methods employed here produced good results which we are moderately confident in.
\end{abstract}

\section{Introduction ADD SOME STUFF HERE GARRETT}
Something about substitution matrices and what they are used for. HMM and the definitions for emission and transition. In this paper we implement two algorithms: Generating a BLOSUM-like substitution matrix using a provided set of 200 sequences and calculating the emission and transition probabilities given a set of sequences that are annotated to indicate the locations where specific amino acids within the gene contribute to different characteristic of the \textit{Silacus Soulas} insect.

\section{Algorithm Descriptions}

\subsection{Substitution Matrix}

\subsection{Emission and Transition Probabilities}
Emission and transition probabilities may be described as the following:

\begin{description}
    \item[Emission] a symbol occurring(emitted) in a genome while in a specific state.
    \item[Transition] a change from one state to another, representing a change in the function or type of the sequence.
\end{description}

In this case, a symbol refers to one of the 20 amino acids, represented by its single letter substitute. Examples can be seen in emission table \ref{emissions}. In the emission parameters are calculated individually for each symbol based on state. The equation used for these calculations is listed below. The term, $e_{s_{x}symbol_y}$, represents the number of occurrences of $symbol_y$ in state $s_x$. This term is divided by all occurrences of any symbol in state $s_x$ to produce a probability of seeing $symbol_y$ in $s_x$.

\begin{equation}
    \frac{e_{s_{x}symbol_y}}{\sum_{k=1}^n e_{s_xsymbol_k}}
\end{equation}


Transition parameters are described as the probability of ``transitioning from state \verb+X+ to state \verb+Y+''(see table \ref{transitions}). They are calculated as the number of observed transitions between two states, $s_x$ and $s_y$, divided by all observed transitions out of state $s_x$. Transitions between the symbols in the same state are counted to estimate the probability of remaining in the current state. The equation for this calculation is more formally stated below with $t$ representing observed transitions between two states.

\begin{equation}
    \frac{t_{s_{x}s_{y}}}{\sum_{k = 0}^{n} t_{s_{x}s_{k}}}
\end{equation}

As may be expected, all probabilities sum to 1, though in slightly different ways for each table. For the emission table, all probabilities for symbols seen in state \verb+0+(seen as one column of the table), for example will sum to 1. For the transition table, however, one horizontal row should be expected to sum to 1. Also note that though some transition probabilities are 0, this is considered acceptable since it is this represents that some segments of genome do not occur following others.

\section{Results}

\begin{table}[!hbp]
    \centering
    \caption{Emission Percentage By State}
    \vspace{0.5cm}
    \begin{tabular}{|c|c  c  c|}
        \hline
        \textbf{Amino Acid} & \multicolumn{3}{|c|}{\textbf{State}} \\
        \hline
        -  &  0 & 1 & 2 \\
        \hline
        a & 3.91 & 7.94 & 1.59 \\
        \hline
        c & 3.84 & 2.75 & 6.86 \\
        \hline
        d & 1.87 & 1.72 & 4.18 \\
        \hline
        e & 3.91 & 2.62 & 5.62 \\
        \hline
        f & 6.60 & 4.96 & 5.55 \\
        \hline
        g & 3.73 & 1.85 & 4.86 \\
        \hline
        h & 3.65 & 1.77 & 3.63 \\
        \hline
        i & 2.91 & 2.13 & 5.01 \\
        \hline
        k & 3.29 & 2.40 & 4.96 \\
        \hline
        l & 3.48 & 5.26 & 4.22 \\
        \hline
        m & 4.67 & 4.28 & 4.79 \\
        \hline
        n & 5.58 & 2.35 & 5.41 \\
        \hline
        p & 5.84 & 3.71 & 5.75 \\
        \hline
        q & 5.44 & 2.18 & 5.38 \\
        \hline
        r & 2.82 & 1.75 & 5.18 \\
        \hline
        s & 5.77 & 4.17 & 4.90 \\
        \hline
        t & 5.96 & 11.10 & 5.75 \\
        \hline
        v & 11.35 & 14.54 & 5.38 \\
        \hline
        w & 7.43 & 11.86 & 6.43 \\
        \hline
        y & 7.97 & 10.66 & 4.56 \\
        \hline
    \end{tabular}
    \label{emissions}
\end{table}


\begin{table}[!hbp]
    \centering
    \caption{Transitions Percentage Between States }
    \vspace{0.5cm}
    \begin{tabular}{|c|c c c|}
        \hline
        \textbf{From State} & \multicolumn{3}{c|}{\textbf{To State}} \\
        \hline
        - & 0 & 1 & 2 \\
        \hline
        0 & 96.1 & 3.89 & 0.00 \\
        1 & 0.00 & 92.01 & 7.99 \\
        2 & 1.54 & 0.00 & 98.5 \\
        \hline
    \end{tabular}
    \label{transitions}
\end{table}

\end{document}
