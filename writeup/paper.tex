\documentclass[a4paper,11pt]{article}
\usepackage[margin=1in]{geometry}
\usepackage{graphicx}
\usepackage{caption}
\usepackage{subcaption}
\usepackage{parskip}

\title{3: Project Write-up}
\author{
Jacob Friedberg \\
\texttt{frie2142@vandals.uidaho.edu}
\and
Garrett Wells \\
\texttt{well1157@vandals.uidaho.edu}
}


\begin{document}
\maketitle
\begin{abstract}
    \noindent This project used techniques presented in the course to estimate parameters for a Hidden Markov Model of the provided data set. We prepared a substitution matrix, emission probability table, and state transition table from the data. The methods employed here produced good results which we are moderately confident in.
\end{abstract}

\section{Introduction}

In bio-informatics one of the crucial tools used to determine how similar two sequences are is a substitution or scoring matrix. These matrices give us a score when aligning sequences that can give us an idea of how related one sequence is to another. Two sequences are aligned with each other and character by character a score is accumulated across the sequence. There are several commonly used substitution matrices including BLOSUM-50, BLOSUM-62, and PAM. To generate these matrices, it is typical calculate the log-odds score for each cell of the matrix. This score is creaded by analyzing  the frequency of at which a pair of characters appears in a large data set of sequences compared to the background frequency of each character and taking the $\log$ of that value. A scaling factor is used to scale the values to the range of $[-10,10]$. 


The purpose behind collecting this information is to model the probability of observing gene sequences which contribute to characteristics of interest. Such a model may then be used to analyze new gene samples to predict which characteristics they correspond to without employing the skills of a highly trained biologist. The model generated with this data is called a Hidden Markov Model(HMM) and uses the sequence of amino acids to predict the most probable state(genomic characteristic) that the sequence encodes for. The statistical information HMMs rely on is defined below:

\begin{description}
    \item[Emission] a symbol occurring(emitted) in a genome while in a specific state.
    \item[Transition] a change from one state to another, representing a change in the function or type of the sequence.
\end{description}

This information is then used to produce a state machine which relies on emissions and transitions to guide the model's state prediction. The algorithms used to calculate these key pieces of data and tables are described below.

In this paper we implement two algorithms: Generating a BLOSUM-like substitution matrix using a provided set of 200 sequences and calculating the emission and transition probabilities given a set of sequences that are annotated to indicate the locations where specific amino acids within the gene contribute to different characteristics of the \textit{Silacus Soulas} insect.


\section{Algorithm Descriptions}

\subsection{Substitution Matrix}





\subsection{Emission and Transition Probabilities}

As introduced in the definition for emission, a symbol refers to one of the 20 amino acids, represented by its single letter substitute. Examples can be seen in emission table \ref{emissions}. In the emission parameters are calculated individually for each symbol based on state. The equation used for these calculations is listed below. The term, $e_{s_{x}symbol_y}$, represents the number of occurrences of $symbol_y$ in state $s_x$. This term is divided by all occurrences of any symbol in state $s_x$ to produce a probability of seeing $symbol_y$ in $s_x$.

\begin{equation}
    \frac{e_{s_{x}symbol_y}}{\sum_{k=1}^n e_{s_xsymbol_k}}
\end{equation}


Transition parameters are described as the probability of ``transitioning from state \verb+X+ to state \verb+Y+''(see table \ref{transitions}). They are calculated as the number of observed transitions between two states, $s_x$ and $s_y$, divided by all observed transitions out of state $s_x$. Transitions between the symbols in the same state are counted to estimate the probability of remaining in the current state. The equation for this calculation is more formally stated below with $t$ representing observed transitions between two states.

\begin{equation}
    \frac{t_{s_{x}s_{y}}}{\sum_{k = 0}^{n} t_{s_{x}s_{k}}}
\end{equation}

As may be expected, all probabilities sum to 1, though in slightly different ways for each table. For the emission table, all probabilities for symbols seen in state \verb+0+(seen as one column of the table), for example will sum to 1. For the transition table, however, one horizontal row should be expected to sum to 1. Also note that though some transition probabilities are 0, this is considered acceptable since it is this represents that some segments of genome do not occur following others.

\section{Results}

\begin{table}[!hbp]
	\centering
	\begin{tabular}{l|*{20}{r} }
		& a  & r  & n & d & c & q  & e  & g  & h & i  & l  & k  & m  & f  & p  & s  & t & w & y  & v \\ \hline
		a & 7  & -1 & 1 & 3 & 2 & -1 & -1 & -2 & 1 & -2 & -2 & -1 & -1 & -2 & -1 & -2 & 3 & 1 & -1 & 1 \\
		r & -1 & 4  & 2 & 1 & 1 & 1  & 2  & 1  & 2 & 1  & 1  & 2  & 3  & 2  & 2  & 2  & 1 & 1 & 1  & 1 \\
		n & 1  & 2  & 4 & 1 & 1 & 2  & 2  & 1  & 2 & 2  & 1  & 1  & 2  & 1  & 3  & 1  & 2 & 1 & 1  & 2 \\
		d & 3  & 1  & 1 & 3 & 2 & 1  & 2  & 2  & 2 & 2  & 1  & 2  & 1  & 2  & 1  & 2  & 2 & 1 & 1  & 2 \\
		c & 2  & 1  & 1 & 2 & 4 & 0  & 0  & 2  & 1 & 1  & 2  & 2  & 2  & 1  & 2  & 2  & 0 & 0 & 3  & 1 \\
		q & -1 & 1  & 2 & 1 & 0 & 4  & 1  & 0  & 2 & 2  & 3  & 1  & 2  & 2  & 1  & 1  & 2 & 1 & 1  & 0 \\
		e & -1 & 2  & 2 & 2 & 0 & 1  & 4  & 2  & 3 & 3  & 0  & 1  & 2  & 1  & 0  & 2  & 1 & 2 & 1  & 0 \\
		g & -2 & 1  & 1 & 2 & 2 & 0  & 2  & 4  & 1 & 2  & 2  & 1  & 1  & 2  & 2  & 3  & 1 & 2 & 0  & 1 \\
		h & 1  & 2  & 2 & 2 & 1 & 2  & 3  & 1  & 4 & 2  & 1  & 2  & 1  & 1  & 0  & 2  & 3 & 1 & 2  & 0 \\
		i & -2 & 1  & 2 & 2 & 1 & 2  & 3  & 2  & 2 & 3  & 1  & 2  & 3  & 3  & 2  & 2  & 1 & 1 & 1  & 0 \\
		l & -2 & 1  & 1 & 1 & 2 & 3  & 0  & 2  & 1 & 1  & 5  & 1  & 1  & 1  & 2  & 2  & 1 & 1 & 2  & 1 \\
		k & -1 & 2  & 1 & 2 & 2 & 1  & 1  & 1  & 2 & 2  & 1  & 3  & 1  & 0  & 2  & 1  & 3 & 2 & 2  & 1 \\
		m & -1 & 3  & 2 & 1 & 2 & 2  & 2  & 1  & 1 & 3  & 1  & 1  & 3  & 2  & 1  & 1  & 1 & 2 & 2  & 1 \\
		f & -2 & 2  & 1 & 2 & 1 & 2  & 1  & 2  & 1 & 3  & 1  & 0  & 2  & 3  & 2  & 1  & 1 & 1 & 1  & 1 \\
		p & -1 & 2  & 3 & 1 & 2 & 1  & 0  & 2  & 0 & 2  & 2  & 2  & 1  & 2  & 4  & 2  & 0 & 1 & 1  & 1 \\
		s & -2 & 2  & 1 & 2 & 2 & 1  & 2  & 3  & 2 & 2  & 2  & 1  & 1  & 1  & 2  & 3  & 2 & 1 & 1  & 2 \\
		t & 3  & 1  & 2 & 2 & 0 & 2  & 1  & 1  & 3 & 1  & 1  & 3  & 1  & 1  & 0  & 2  & 3 & 1 & 2  & 1 \\
		w & 1  & 1  & 1 & 1 & 0 & 1  & 2  & 2  & 1 & 1  & 1  & 2  & 2  & 1  & 1  & 1  & 1 & 3 & 0  & 2 \\
		y & -1 & 1  & 1 & 1 & 3 & 1  & 1  & 0  & 2 & 1  & 2  & 2  & 2  & 1  & 1  & 1  & 2 & 0 & 4  & 1 \\
		v & 1  & 1  & 2 & 2 & 1 & 0  & 0  & 1  & 0 & 0  & 1  & 1  & 1  & 1  & 1  & 2  & 1 & 2 & 1  & 4\\
		\hline
	\end{tabular}
	\caption{Generated BLOSUM-like scoring matrix}
    \label{substitutions}
\end{table}

\begin{table}[!hbp]
    \centering
    \caption{Emission Percentage By State}
    \vspace{0.5cm}
    \begin{tabular}{|c|c  c  c|}
        \hline
        \textbf{Amino Acid} & \multicolumn{3}{|c|}{\textbf{State}} \\
        \hline
        -  &  0 & 1 & 2 \\
        \hline
        a & 3.91 & 7.94 & 1.59 \\
        \hline
        c & 3.84 & 2.75 & 6.86 \\
        \hline
        d & 1.87 & 1.72 & 4.18 \\
        \hline
        e & 3.91 & 2.62 & 5.62 \\
        \hline
        f & 6.60 & 4.96 & 5.55 \\
        \hline
        g & 3.73 & 1.85 & 4.86 \\
        \hline
        h & 3.65 & 1.77 & 3.63 \\
        \hline
        i & 2.91 & 2.13 & 5.01 \\
        \hline
        k & 3.29 & 2.40 & 4.96 \\
        \hline
        l & 3.48 & 5.26 & 4.22 \\
        \hline
        m & 4.67 & 4.28 & 4.79 \\
        \hline
        n & 5.58 & 2.35 & 5.41 \\
        \hline
        p & 5.84 & 3.71 & 5.75 \\
        \hline
        q & 5.44 & 2.18 & 5.38 \\
        \hline
        r & 2.82 & 1.75 & 5.18 \\
        \hline
        s & 5.77 & 4.17 & 4.90 \\
        \hline
        t & 5.96 & 11.10 & 5.75 \\
        \hline
        v & 11.35 & 14.54 & 5.38 \\
        \hline
        w & 7.43 & 11.86 & 6.43 \\
        \hline
        y & 7.97 & 10.66 & 4.56 \\
        \hline
    \end{tabular}
    \label{emissions}
\end{table}


\begin{table}[!hbp]
    \centering
    \caption{Transitions Percentage Between States }
    \vspace{0.5cm}
    \begin{tabular}{|c|c c c|}
        \hline
        \textbf{From State} & \multicolumn{3}{c|}{\textbf{To State}} \\
        \hline
        - & 0 & 1 & 2 \\
        \hline
        0 & 96.1 & 3.89 & 0.00 \\
        1 & 0.00 & 92.01 & 7.99 \\
        2 & 1.54 & 0.00 & 98.5 \\
        \hline
    \end{tabular}
    \label{transitions}
\end{table}

\end{document}
